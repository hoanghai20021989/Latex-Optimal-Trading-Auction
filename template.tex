\documentclass{article}


\usepackage{arxiv}

\usepackage[utf8]{inputenc} % allow utf-8 input
\usepackage[T1]{fontenc}    % use 8-bit T1 fonts
\usepackage{hyperref}       % hyperlinks
\usepackage{url}            % simple URL typesetting
\usepackage{booktabs}       % professional-quality tables
\usepackage{amsfonts}       % blackboard math symbols
\usepackage{nicefrac}       % compact symbols for 1/2, etc.
\usepackage{microtype}      % microtypography

% Packages that are not related to the template
\usepackage{natbib}
\usepackage{graphicx}
\usepackage{amsthm}
\usepackage{amsmath}
\usepackage{systeme}
\usepackage{algorithmic}
\usepackage[thinc]{esdiff}

\graphicspath{ {./images/} }

\newtheorem{definition}{Definition}

\newcommand{\numberOfStocks}{350}

\title{Optimal liquidation strategy in a mixed auction market}

\author{
 Hoang Hai Tran\\
 Department of Statistics \& Applied Probability\\
 National University of Singapore \\
 Lower Kent Ridge Road 10 \\
 119076 Singapore \\
 \texttt{e0045275@nus.edu.sg} \\
   \And
 Ying Chen\\
 Department of Mathematics and Risk Management Institute \\
 National University of Singapore \\
 Lower Kent Ridge Road 10 \\
 119076 Singapore \\
 \texttt{matcheny@nus.edu.sg} \\
}

\begin{document}
\maketitle

\begin{abstract}
  We analyze the price impact of trading in a mixed auction market, in which the exchange employs both call auction and continuous auction. Based on a multiplayer game settings between a market maker and one or more liquidity takers, we derive a market impact model which characterizes both instantaneous and temporary price impact. The unified model is able to show how price discovery and market impact transition from the call auction to the continuous auction. Moreover, we provide an optimal trading strategy with closed-form solution to execute a large order under both auctions. The optimal trading strategy is numerically demonstrated with the cost efficiency of trading in both auctions compared to alternative only in the continuous trading auction.
\end{abstract}

\section{Appendix}

\bibliographystyle{plain}
\bibliography{references}
\end{document}

