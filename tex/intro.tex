\section{Introduction}
In modern financial markets, double auction is a standard process of buying and selling shares adopted by exchanges. From an economist's perspective, an auction is a price discovery process where a competitive equilibrium price is found through interactions among bidders. The question of how the price reaches its equilibrium points, and how different players use their own information advantages to achieve a preferable price, has attracted attentions from various fields. \cite{Kyle1985} studies how market makers, insider traders and noise traders interact to form an equilibrium price in a continuous trading auction. \cite{Almgren2000} discusses how execution can be done optimally in a continuous trading phase under certain market impact assumptions. \cite{Avellaneda2008} derives an equilibrium price from the perspectives of a market maker to provide quotes in a continuous trading auction. This paper is in a similar vein, in which we discuss how price will reach its equilibrium price in a mixed auction market. By taking into account call auction, under different market impact assumptions, we derive the corresponding optimal liquidation strategy for a trader to unwind his position. We provide theoretical evidence that it is indeed cheaper to trade the call auction compared to the continuous auction. Specifically, we examine the market impact of trading in a market that operates under two auction models:

\begin{itemize}
  \item \textbf{Call auction} Buyers set a maximum price that they want to buy, and sellers set a minimum price that they want to sell. The orders will then be batched together and matched at a price set by the exchange.
  \item \textbf{Continuous auction} Orders are matched continuously based on time and price priority. Participants in the auction can choose to either queue their orders to wait for an opposing trade, or cross the book to match against other's orders.
\end{itemize}

We seek to reduce market impact and cost of trading by utilizing both auctions in our liquidation algorithm. The aim is to significantly reduce the cost of trading by exploiting liquidity from both, which can be achieved from answering the following questions:

\begin{itemize}
  \item What are the relevant factors that affect market impact under the mixed auction scenario?
  \item How does the transition of market impact from a call auction into a continuous auction evolve?
  \item How can we optimize for a liquidation trading strategy, taken into account those factors?
\end{itemize}

In general, the challenge lies in the different mechanisms of the auctions, and the modelling of market impact during both thereof. By modelling the interaction between different participants in the market, namely between market makers and liquidity takers, we derive the mechanism of market impact under both auctions. From the mechanism, we optimize for a liquidation problem that results in on average 50\% reduction empirically in transaction cost against the classical TWAP solution during continuous trading phase only under a risk-neutral settings. {\color{red}Furthermore, we show that the trader will be more willing to increase his trading size during auction if he expects the volatility of the day to be higher, and/or being more risk-averse. This highlights another benefit of incorporating auction besides cost savings: variance reduction of the overall portfolio exposure.}

It is not a coincidence that by incorporating call auction into the optimal liquidation problem, we can reduce both {\color{red}cost and variance} significantly. Historically, the adoption of the call auctions has resulted in a substantial shift of daily trading activity, with the call auctions responsible for a significant proportion of the average daily trading volumes. (\cite{Madhavan2015}) found that the call auction in New York Stock Exchange (NYSE) accounts for roughly 10 percent of the average trade volume. (\cite{Bruno1999}) found similar phenomenon on the Paris Bourse, in which the opening auction accounts for about 10 percent of average daily trade value. (\cite{Carole2006}) has a similar conclusion regarding the Australian Stock Exchange (ASX).

A call auction can lower execution costs for participants and improve price discovery process, as suggested in (\cite{Pagano2003}). The reduction in trading costs, such as market impact and adverse selection costs, {\color{red}and the volatility dampening effect are} achieved through several channels, including (\cite{Carole2006}):

\begin{itemize}
  \item Temporal consolidation of order flow and execution of orders at a single price (\cite{Economides1995}).
  \item Facilitation of price discovery process during times of market stress (\cite{Madhavan1992}).
\end{itemize}

Various studies have found that there are significant strategic trading activities during the call auctions. (\cite{Bruno1999}) suggests that the fluctuation in the indicative price during the call auction period maybe due to strategic trading. (\cite{Vives2001}) argues that large market makers may have incentives to trade during the opening call auction, to mitigate information leakage.

On the other hand, literature on optimal liquidation and market impact has rarely mentioned execution during call auctions, not to mention the mixed auction environment. These works are normally derived for continuous auction. One of the first study on optimal liquidation is (\cite{Ho1981}), in which an optimal trading curve is derived under certain assumptions on stochastic trade arrival and price process. (\cite{Almgren2000}) extends the model further by incorporating a trader's risk aversion factor when pricing for an optimal liquidation strategy. (\cite{Obizhaeva2013}) solves the problem under a different assumption: the impact of each trade recovers slowly instead of immediately. In terms of trading during a call auction, (\cite{Madhavan2015}) derives a optimal liquidation strategy for a specialist to set an equilibrium price under the assumptions of multiple traders with private information.

Works on optimal execution today focuses more on minimizing the impact of trading during the continuous trading phase, which ignores the fact that the opening auction is usually very liquid and the matched volume represents a significant portion of daily trading volume. For executing a big block order with significant volume, including call auctions are usually more preferable since the execution cost is lower, as shown in our analysis later. Specifically, based on our theoretical framework, we found that:

\begin{itemize}
  \item The call auction provides another venue to take liquidity from.
  \item Due to inventory management from the market maker, the transition period from the call auction will be characterized with a wide initial spread that will narrow over time. A mixed auction strategy takes into account this effect will reduce transition cost even further.
  \item {\color{red} The immediate matching feature of the call auction allows significant volatility reduction of the trading strategy when risk aversion is accounted for}
  \item There is an optimal liquidation strategy to execute trade during both the call auction and the continuous auction that results in better cost optimization than trading in the continuous auction only.
\end{itemize}

This paper is organized as follows: Section (\ref{sec:AnalyticalFramework}) will present the theoretical framework that we will use to characterize the call auction and optimize our trading strategy. Section (\ref{sec:EmpiricalAnalysis}) will perform empirical analysis based on the theoretical framework to examine their effects on the optimal liquidation strategy and its cost efficiency. Section (\ref{secConclusionAuction}) will conclude the paper with suggestions for further research.